\documentclass[lang=cn,11pt]{elegantpaper}
\title{预备工作2 : 定义编译器功能\&汇编语言编程}
\author{高文畅 1812983}
\date{}

\begin{document}
	\maketitle
	\begin{abstract}
		本次实验分析了GCC编译器支持的C语言特性,在此基础上定义了C语言的子集,之后用上下文无关文法对其做了简单的描述.之后以一个简单的C语言程序为例,该程序用来求解较小范围的整数n的阶乘,编写了与之等价的x86汇编程序.
	\end{abstract}
	\keywords{编译系统, C语言, 上下文无关文法, 汇编语言}
	\newpage
	\tableofcontents
	\newpage
	
	\section{编译器支持的C语言特性及C语言子集}
	\subsection{支持的特性}
	\subsubsection{数据类型}
	基本数据类型: short, int, long, char, float, double, bool \par
	指针(如int* 等), 结构体(struct),枚举{enum},联合{union}
	\subsubsection{变量,常量的声明和赋值}
	\subsubsection{定义函数}
	\subsubsection{注释}
	\subsubsection{分支结构}
	分支结构包括关键字: if, else, switch, case, default
	\subsubsection{循环结构}
	循环结构包括关键字: for, while, do
	\subsection{用CFG描述C语言子集}
	\subsubsection{标识符的命名}
	\begin{equation*}
		\begin{split}
			&letter \rightarrow a\ |\ b\ |...|\ z\ |\ A\ |\ B\ |...|\ Z \\
			&digit \rightarrow 0\ |\ 1\ |...|\ 9 \\
			&begin \rightarrow \_\ |\ \epsilon \\
			&body \rightarrow body\ letter\ |\ body\ digit\ |\ letter\ |\ digit\ |\ \_\ |\ \epsilon \\
			&id \rightarrow beginletterbody \\
		\end{split}
	\end{equation*}
	\subsubsection{变量声明}
	\begin{equation*}
		\begin{split}
		&type \rightarrow *type\ |\ short\ |\ int\ |\ long\ |\ char\ |\ float\ |\ double\ |\ bool \\
		&ids \rightarrow ids,\ id\ |\ id \\
		&var\_decl \rightarrow type\ ids \\
		\end{split}	
	\end{equation*}
	\subsubsection{变量赋值}
	\begin{equation*}
		\begin{split}
			&integer \rightarrow integer\ digit\ |\ digit \\
			&float \rightarrow integer.integer\ |\ integer \\
			&boolean \rightarrow true\ |\ false\ \\
			&value \rightarrow integer\ |\ integer\ |\ float\ |\ boolean \\
			&var\_assi \rightarrow id\ =\ integer\ |\ float\ |\ boolean\ \\
		\end{split}
	\end{equation*}
	\subsection{数组声明}
	\begin{equation*}
		\begin{split}
			&decl \rightarrow type\ arrays \\
			&arrays \rightarrow arrays,\ id[integer]\ |\ id[integer] \\
		\end{split}
	\end{equation*}
	\subsection{函数定义}
		\begin{equation*}
		\begin{split}
			&ftype \rightarrow type\ |\ void \\
			&fun\_defn \rightarrow ftype\ funname(paras)\ \{stmt\} \\
			&paras \rightarrow paras,\ type\ id\ |\ type\ id\ |\ \epsilon \\
			&funname \rightarrow id \\
		\end{split}
	\end{equation*}
	\subsection{循环语句}
		\begin{equation*}
			\begin{split}
				&stmt \rightarrow while(expr)\ stmt \\
				&stmt \rightarrow for(expr;expr;expr) stmt \\
				&stmt \rightarrow do\ stmt while(expr) \\
			\end{split}
		\end{equation*}
	\subsection{条件语句}
		\begin{equation*}
		\begin{split}
			&stmt \rightarrow if(stmt) else\ stmt \\
			&stmt \rightarrow if(stmt)\ stmt \\
		\end{split}
		\end{equation*}
	\section{汇编程序}
	\subsection{原C语言程序}
	\begin{lstlisting}
#define _CRT_SECURE_NO_WARNINGS
#include <stdio.h>
#define INF 9999
int x =101;
int fact(int a)
{
	int i=1,ans=1;
	for(;i<=a;i++)
	ans*=i;
	return ans;
}
int main()
{
	int n;
	scanf("%d",&n);
	printf("%d\n",fact(n));
	printf("%d",x);
	return 0;
}
	\end{lstlisting}
	\subsection{汇编语言程序}
	\begin{lstlisting}
		    .text
		.global max
		.type fact, @function
		max:
		movl    8(%ebp), %edx
		movl    $1, %eax
		cmpl    $1, %edx
		jle     L1
		
		L0:
		imull   %edx, %eax
		subl    $1, %edx
		cmpl    $1, %edx
		jg  .L0
		
		L1:         done:
		x: 
		.zero 4
		.align 4
		
		
		.section    .rodata
		STR0:
		.string "%d"
		STR1:
		.string "%d\n"
		STR2:
		.string "%d"
		
		
		.text
		.gloabal    main
		.type main, @function
		main:
		pushl   $n
		pushl   $STR0
		call    scanf
		addl    $12, esp
		movel   n, %rbp+8
		pushl   %eax
		pushl   $STR1
		call    printf
		movl    x, %eax
		pushl   %eax
		pushl   $STR2
		call    printf
		addl    $8, %esp
		movl    $0, %eax
		ret
		
		.section    .note.GNU-stack,"",@progbits
	\end{lstlisting}
	\subsection{用gcc生成的汇编语言程序}
		\begin{lstlisting}
	.file	"E2.c"
.text
.globl	x
.data
.align 4
x:
.long	100
.text
.globl	max
.def	max;	.scl	2;	.type	32;	.endef
.seh_proc	max
max:
pushq	%rbp
.seh_pushreg	%rbp
movq	%rsp, %rbp
.seh_setframe	%rbp, 0
subq	$16, %rsp
.seh_stackalloc	16

.seh_endprologue
movq	%rcx, 16(%rbp)
movl	%edx, 24(%rbp)
movl	$0, -4(%rbp)
movl	$-9999, -8(%rbp)
jmp	.L2
.L4:
movl	-4(%rbp), %eax
cltq
leaq	0(,%rax,4), %rdx
movq	16(%rbp), %rax
addq	%rdx, %rax
movl	(%rax), %eax
cmpl	%eax, -8(%rbp)
jge	.L3
movl	-4(%rbp), %eax
cltq
leaq	0(,%rax,4), %rdx
movq	16(%rbp), %rax
addq	%rdx, %rax
movl	(%rax), %eax
movl	%eax, -8(%rbp)
.L3:
addl	$1, -4(%rbp)
.L2:
movl	-4(%rbp), %eax
cmpl	24(%rbp), %eax
jl	.L4
movl	-8(%rbp), %eax
addq	$16, %rsp
popq	%rbp
ret
.seh_endproc
.def	__main;	.scl	2;	.type	32;	.endef
.section .rdata,"dr"
.LC0:
.ascii "%d\0"
.text
.globl	main
.def	main;	.scl	2;	.type	32;	.endef
.seh_proc	main
main:
pushq	%rbp
.seh_pushreg	%rbp
movq	%rsp, %rbp
.seh_setframe	%rbp, 0
subq	$96, %rsp
.seh_stackalloc	96
.seh_endprologue
call	__main
movl	$0, -4(%rbp)
leaq	-52(%rbp), %rax
movq	%rax, %rdx
leaq	.LC0(%rip), %rcx
call	scanf
jmp	.L7
.L8:
movl	-52(%rbp), %eax
imull	-4(%rbp), %eax
movl	%eax, %edx
movl	-4(%rbp), %eax
cltq
movl	%edx, -48(%rbp,%rax,4)
addl	$1, -4(%rbp)
.L7:
cmpl	$9, -4(%rbp)
jle	.L8
leaq	-48(%rbp), %rax
movl	$10, %edx
movq	%rax, %rcx
call	max
movl	%eax, -8(%rbp)
movl	-8(%rbp), %eax
movl	%eax, %edx
leaq	.LC0(%rip), %rcx
call	printf
movl	$0, %eax
addq	$96, %rsp
popq	%rbp
ret
.seh_endproc
.ident	"GCC: (x86_64-posix-seh-rev0, Built by MinGW-W64 project) 7.3.0"
.def	scanf;	.scl	2;	.type	32;	.endef
.def	printf;	.scl	2;	.type	32;	.endef
+
		\end{lstlisting}
	
\end{document}